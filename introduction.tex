
% disable for numbered page 
\thispagestyle{empty}
%
\chapter{Introduction to general \LaTeX{} features}
\label{chap:introduction}

A possible way to start a chapter/thesis is a matching quote such as this:

\begin{quote}
  "Design for Reliability is nothing but Design for yield in the
  temporal dimension"\\
  Giovanni De Micheli 2004, EPF Lausanne
\end{quote}

The following sections are dedicated to various features \LaTeX{} offers to the author.

\section{Listing Environments}
\label{sec:listingenvironments}

Itemize:
\begin{itemize}
	\item Information redundancy: check sums, state encoding, etc.
	\item Spatial redundancy: TMR
	\item Temporal redundancy: check points, repeated calculations
\end{itemize}

\medskip

Enumerate:
\begin{enumerate}
	\item Information redundancy: check sums, state encoding, etc.
	\item Spatial redundancy: TMR
	\item Temporal redundancy: check points, repeated calculations
\end{enumerate}

\medskip

Description:
\begin{description}
	\item[Information redundancy:] check sums, state encoding, etc.
	\item[Spatial redundancy:] TMR
	\item[Temporal redundancy:] check points, repeated calculations
\end{description}

\section{An overview of pre-defined text formats}
\label{sec:overview_textformats}

\setstretch{1.0}
\verb$\emph{}:$ \emph{Lorem ipsum dolor sit amet, consectetur adipisici elit.}\\
\verb$\textbf{}:$ \textbf{Lorem ipsum dolor sit amet, consectetur adipisici elit.}\\
\verb$\textit{}:$ \textit{Lorem ipsum dolor sit amet, consectetur adipisici elit.}\\
\verb$\textsl{}:$\footnote{italic and slanted are different fonts!} \textsl{Lorem ipsum dolor sit amet, consectetur adipisici elit.}\\
\verb$\textsf{}:$ \textsf{Lorem ipsum dolor sit amet, consectetur adipisici elit.}\\
\verb$\textsc{}:$ \textsc{Lorem ipsum dolor sit amet, consectetur adipisici elit.}\\
\verb$\texttt{}:$ \texttt{Lorem ipsum dolor sit amet, consectetur adipisici elit.}\\
\verb&\verb$$:& \verb$Lorem ipsum dolor sit amet, consectetur adipisici elit.$\\
\verb$\large{}:$ \large{Lorem ipsum dolor sit amet, consectetur adipisici elit.}\\
\verb$\Large{}:$ \Large{Lorem ipsum dolor sit amet, consectetur adipi\dots}\\
\verb$\LARGE{}:$ \LARGE{Lorem ipsum dolor sit amet, consecte\dots}\\
\verb$\huge{}:$ \huge{Lorem ipsum dolor sit amet \dots}\\
\verb$\Huge{}:$ \Huge{Lorem ipsum dolor sit amet \dots}
\normalsize{}

Text formatting can be reset to normal via the :
\verb$\normalsize{}$
command.

Footnotes can be set via the \verb$\footnote{}$ command.\\
Here is an example:
\verb$\footnote{Creating footnotes is easy!}$ will translate into this.\footnote{Creating footnotes is easy!}

\section{Symbols in \LaTeX}
\label{sec:symbols}

\begin{table}[H] %[H] means place here under all circumstances!
\centering{
	% Note that the caption appears before the actual table
	% --> will be correctly placed above it!
	\caption{List of commonly used \LaTeX{} symbols/characters}
	\label{tab:specialchars}
	\vspace{0.5cm}
	\begin{tabular}{ l l l }
		\toprule
		\textbf{Symbol/Character}	& \textbf{\LaTeX{} source code}	& \textbf{Comments}\\
		\midrule
		Space			         	& \verb$~$              & Avoid if possible\dots \\
		-                   & \verb$-$              & Short hyphen/dash (O-Beine)\\
		--                  & \verb$--$             & Long hyphen/dash (8 a.m.-- 6 p.m.)\\
		---                 & \verb$---$            & Long english hyphen/dash\\ 
		\dq Text\dq\        & \verb$\dq Text \dq$   & English quotation marks\\ 
		\glqq Text\grqq\    & \verb$\glqq ...\grqq$ & German quotation marks\\ 
		\glq Text\grq\      & \verb$\glq ...\grq$   & German, simple apostrophe\\ 
		\EUR{50}            & \verb$\EUR{50}$       & Euro symbol\\ 
		\o{}                & \verb$\o{}$           & Diameter\\ 
		\$                  & \verb&\$&             & Dollar sign\\ 
		\%                  & \verb$\%$             & Per cent sign \\ 
		\&                  & \verb$\&$             & Ampersand \\ 
		\#                  & \verb$\#$             & Hash\\ 
		\{                  & \verb$\{$             & Left brace\\ 
		\}                  & \verb$\}$             & Right brace\\ 
		\_                  & \verb$\_$             & Underscore\\ 
		\S                  & \verb$\S$             & Paragraph\\ 
		\copyright          & \verb$\copyright$     & Copyright\\ 
		\pounds             & \verb$\pounds$        & British Pound\\ 
		\dots               & \verb$\dots$          & Tripple dots\\ 
		$\cdots$            & \verb&$\cdots$&       & Tripple dots, height-centered\\ 
		$\times$            & \verb&$\times$&       & Multiplication\\ 
		$\backslash$        & \verb&$\backslash$&   & Backslash, italic (math mode)\\ 
		\textbackslash      & \verb$\textbackslash$ & Backslash\\ 
		$\leftarrow$        & \verb&$\leftarrow$&   & \\ 
		$\rightarrow$       & \verb&$\rightarrow$&  & \\ 
		$\uparrow$          & \verb&$\uparrow$&     & \\ 
		$\downarrow$        & \verb&$\downarrow$&   & \\ 
		$\Leftarrow$        & \verb&$\Leftarrow$&   & \\ 
		$\Rightarrow$       & \verb&$\Rightarrow$&  & \\ 
		$\Uparrow$          & \verb&$\Uparrow$&     & \\ 
		$\Downarrow$        & \verb&$\Downarrow$&   & \\ 
		$\hookleftarrow$    & \verb&$\hookleftarrow$& & Indicates line break\\ 
		$\pm$               & \verb&$\pm$&          & \\ 
		$\mp$               & \verb&$\mp$&          & \\ 
		\bottomrule
		\end{tabular}
} % \centering
\end{table}

\section{Algorithms \& Pseudo-Code}
\label{sec:math}

Let there be an linear system $A \textbf{x} = \textbf{b}$, with  $A$ in $\mathds{R}^{(n, n)}$, $\textbf{x} \in \mathds{R}^n$ und $\textbf{b} \in \mathds{R}^n$. Additionally let $A$ be regular. One possibility to determine the solution vector $\textbf{x}$ can be found using the \emph{gaussian elimination}. The following illustrates the algorithm employing a column pivot search as pseudo code.

\begin{algorithm}
\caption{SolveLinearSystemGauss($A, \textbf{b}$) $\rightarrow \textbf{x}$}
\label{alg:gauss}
\begin{algorithmic}
{\small
\FOR{$i := 0$ \textbf{to} $n-1$}
  \STATE $max := 0, p := -1$
  \STATE
  \FOR {$j := i$ \textbf{to} $n-1$}
    \IF {$(|a_{ji}|\ >\ max)$}
      \STATE $max := |a_{ji}|, p := j$
    \ENDIF
  \ENDFOR
  \STATE
  \IF {$p = -1$}
    \STATE STOP \COMMENT{Matrix $A$ is not regular}
  \ENDIF
  \STATE
  \IF {$p \neq i$}
    \STATE SwapLines(A, $i$, $p$)
    \STATE $s := b_i, b_i := b_p, b_p := s$
  \ENDIF
  \STATE
  \STATE $pivot := a_{ii}$
  \STATE
  \FOR {$j := i + 1$ \textbf{to} $n-1$}
    \STATE $factor := a_{ji} / pivot$
    \STATE $b_j := b_j - factor \cdot b_i$
    \FOR {$k := i + 1$ \textbf{to} $n-1$}
      \STATE $a_{jk} := a_{jk} - factor \cdot a_{ik}$
    \ENDFOR
  \ENDFOR
\ENDFOR
\STATE
\FOR {$i := n-1$ \textbf{downto} $0$}
  \STATE $sum := 0$
  \FOR {$j := i + 1$ \textbf{to} $n-1$}
    \STATE $sum := sum + a_{ij} \cdot x_j$
  \ENDFOR
  \STATE $x_i := (b_i - sum) / a_{ii}$
\ENDFOR
}
\end{algorithmic}
\end{algorithm}
\clearpage

\setstretch{1.5}

\section{Images \& Figures}
\label{sec:images}

One may include pictures as a single figure or multiple pictures together as one figure using the package $subfigure$.
Generally vector graphics shall be preferred to raster graphics.

This is a single image in a floating figure environment:
\setstretch{1.0}
\begin{verbatim}
\begin{figure}[htbp]
	\centering
	\includegraphics[width=0.8\linewidth]{pics/let_distance}
	\caption{Linear Energy Transfer (LET) in Abhängigkeit der
          Ursprungsenergie und des Ursprungspartikels \cite{baumann2005b}}
	\label{fig:grundlagen_softerrors_letgraph}
\end{figure}
\end{verbatim}

\verb$[htbp]$ indicates placement options for \LaTeX{}. These can be 
\begin{itemize}
	\item \verb$[h]$ \dots{} here
	\item \verb$[t]$ \dots{} top of the page
	\item \verb$[b]$ \dots{} bottom of the page
	\item \verb$[p]$ \dots{} on a seperate page
\end{itemize}

\begin{figure}[htbp]
	\centering
	\includegraphics[width=0.8\linewidth]{pics/let_distance}
	\caption{Linear Energy Transfer (LET) in Abhängigkeit der
          Ursprungsenergie und des Ursprungspartikels \cite{baumann2005b}}
	\label{fig:figure_example}
\end{figure}

\setstretch{1.5}

When there is a need to display smaller images of a common context, one can use \emph{subfigures}.
See the documentation of the $subfigure$ package for further options, especially regarding caption placement.
\begin{figure}[htbp]
\centering
\subfigure[Bestandteile einer fiktiven FPGA-Architektur]{
	\includegraphics[width=0.3\linewidth]{pics/fpga_simple_architecture}
	%\caption{Bestandteile einer fiktiven FPGA-Architektur}
	\label{fig:subfig_fpga}
}
\subfigure[Logischer Aufbau einer LUT mit zwei Eingängen]{
\includegraphics[width=0.3\linewidth]{pics/lut}
\label{fig:subfig_lut}
}
\subfigure[Aufbau eines ``Adaptive Logic Module'' der Stratix\TReg\ II-Familie]{
	\includegraphics[width=0.5\linewidth]{pics/alm_internal}
	\label{fig:subfig_alm}
}
\label{fig:subfigure_example}
\caption[Subfig Example: Abbreviated caption used in list of figures]{Complete Caption within which one may reference the individual subfigures \subref{fig:subfig_fpga}, \subref{fig:subfig_lut} and \subref{fig:subfig_alm} for further description. }
\end{figure}